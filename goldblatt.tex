\documentclass[a4paper, 11pt]{book}
\usepackage{comment} % enables the use of multi-line comments (\ifx \fi) 
\usepackage{lipsum} %This package just generates Lorem Ipsum filler text. 
\usepackage{fullpage} % changes the margin
\usepackage[a4paper, total={7in, 10in}]{geometry}

%\usepackage{tgadventor} % The font for the entire document can be changed here
%\usepackage{courier}
%\usepackage{charter}
\usepackage{tgcursor}

\usepackage{mathrsfs} 
\usepackage{quiver} 
\newtheorem{corollary}{Corollary}
\usepackage{graphicx}
\usepackage{tikz}

\usetikzlibrary{arrows}
\usepackage{verbatim}
\usepackage[numbered]{mcode}
\usepackage{float}
\usepackage{tikz}
    \usetikzlibrary{shapes,arrows}
    \usetikzlibrary{arrows,calc,positioning}

    \tikzset{
        block/.style = {draw, rectangle,
            minimum height=1cm,
            minimum width=1.5cm},
        input/.style = {coordinate,node distance=1cm},
        output/.style = {coordinate,node distance=4cm},
        arrow/.style={draw, -latex,node distance=2cm},
        pinstyle/.style = {pin edge={latex-, black,node distance=2cm}},
        sum/.style = {draw, circle, node distance=1cm},
    }
\usepackage{xcolor}
\usepackage{mdframed}
\usepackage[shortlabels]{enumitem}
\usepackage{indentfirst}
\usepackage{hyperref}
\usepackage{amsmath,amsfonts,amsthm, amssymb}
\usepackage{array}
\usepackage[all,textures]{xy}
\usepackage{graphicx}
\usepackage{alltt}
\usepackage{listings}
\usepackage{float}
\usepackage{tabu}
\usepackage{longtable}


\theoremstyle{plain}
\newtheorem{theorem}{Proposition}[section]
\newtheorem{proposition}[theorem]{Proposition}

\newtheorem{question}[theorem]{Question}
\newtheorem{exercise}{Exercise}[chapter]
\newtheorem{problem}[theorem]{Problem}
    
\renewcommand{\thesubsection}{\thesection.\alph{subsection}}


% Define solution environment
\newenvironment{answer}
    {\textit{answer:}}
    {}
%%%%%%%%%%%%%%%%%%%%%%%%%%%%%%%%%%%%%%%%%%%%%%%%%%%%%%%%%%%%%%%%%%%%%%%%%%%%%%%%%%%%%%%%%%%%%%%%%%%%%%%%%%%%%%%%%%%%%%%%%%%%%%%%%%%%%%%% Original packages, custom environments, and custom commands below
  \usepackage{amsmath,amsfonts,amsthm, amssymb}
  \usepackage{fullpage}
  \usepackage{array}
  \usepackage[all,textures]{xy}
  \usepackage{graphicx}
  \usepackage{alltt}
  \usepackage{listings}
  \usepackage{float}
  \usepackage{tabu}
  \usepackage{longtable}
  \usepackage{lipsum}
  \usepackage[T1]{fontenc}
  
  \theoremstyle{plain}
  \newtheorem{innercustomgeneric}{\customgenericname}
\providecommand{\customgenericname}{}
\newcommand{\newcustomtheorem}[2]{%
  \newenvironment{#1}[1]
  {%
   \renewcommand\customgenericname{#2}%
   \renewcommand\theinnercustomgeneric{##1}%
   \innercustomgeneric
  }
  {\endinnercustomgeneric}
}

\newcustomtheorem{definition}{Definition}
\newcustomtheorem{lemma}{Lemma}

\newcommand{\mb}{\mathbf}
\newcommand{\arr}{\rightarrow}
\newcommand{\mc}{\mathcal}
\newcommand{\ms}{\mathscr}
\newcommand{\co}{\text{co}}
\newcommand{\N}{\mathbb{N}}
\newcommand{\Z}{\mathbb{Z}}
\newcommand{\Q}{\mathbb{Q}}
\newcommand{\R}{\mathbb{R}}
\newcommand{\I}{\mathbb{I}}
\newcommand{\la}{\langle}
\newcommand{\ra}{\rangle}
\newcommand{\op}{\oplus}
\newcommand{\od}{\odot}
\newcommand{\p}{\prime}
\newcommand{\ep}{\varepsilon}
\newcommand{\del}{\delta}
\newcommand{\hyp}{{}^*}
\newcommand{\dom}{\text{dom }}
\newcommand{\hal}{\text{hal}}
\newcommand{\gal}{\text{gal}}
\newcommand{\sh}{\text{sh}}

  \setlength{\parindent}{0pt}
 
 
 \newtheorem{thm}{Exercise}
\begin{document}
  \begin{titlepage}
	\centering % Center everything on the title page
	\scshape % Use small caps for all text on the title page
	\vspace*{1.5\baselineskip} % White space at the top of the page
% ===================
%	Title Section 	
% ===================

	\rule{13cm}{1.6pt}\vspace*{-\baselineskip}\vspace*{2pt} % Thick horizontal rule
	\rule{13cm}{0.4pt} % Thin horizontal rule
	
		\vspace{0.75\baselineskip} % Whitespace above the title
% ========== Title ===============	
	{	\Huge Lectures on Hyperreals\\ 
			\vspace{4mm}
		Book by Robert Goldblatt \\	}
% ======================================
		\vspace{0.75\baselineskip} % Whitespace below the title
	\rule{13cm}{0.4pt}\vspace*{-\baselineskip}\vspace{3.2pt} % Thin horizontal rule
	\rule{13cm}{1.6pt} % Thick horizontal rule
	
		\vspace{1.75\baselineskip} % Whitespace after the title block
% =================
%	Information	
% =================
	{\large : 
		\vspace*{1.2\baselineskip}
	} \\
	\vfill

\end{titlepage}
%%%%%%%%%%%%%%%%%%%%%%%%%%%%%%%%%%%%%%%%%%%%%%%%%%%%%%%%%%%
\begingroup
\let\cleardoublepage\clearpage
\tableofcontents
\endgroup

\chapter{What are Hypereals?}

\newpage

\chapter{Large sets}

  \begin{exercise}
    If $\emptyset \not = A \subseteq I$, there is an ultrafilter $\mc{F}$ on $I$ with $A \in \mc{F}$.
  \end{exercise}
  \begin{proof}
    Define $\mc{H}=\{B \in \mathscr{P}(I): A \subseteq B\}$. Since $A$ is non empty, it follows that $\mc{H}$ has fip. Clearly, $A \in \mc{H}$, and so $A \in \mc{F}^\mc{H}$. Therefore, by zorn lemma, we know that there exists ultrafilter, $\mc{F}$, such that $A \in \mc{F}^\mc{H} \subseteq \mc{F}$.
  \end{proof}

  \begin{exercise}
    There exists a nonprincipal ultrafilter on $\N$ containing the set of even numbers, and another containing the set of odd numbers.
  \end{exercise}
  \begin{proof}
    Define $I=\{x \in \N: x=2n$ for some $n \in \N\}$. Since $I$ is infinite, we know that $\mc{F}^\co$ is proper and thus, have fip. Applying the zorn's lemma, we see that there exists ultrafilter $\mc{F}$ such that $\mc{F} \subseteq \mc{F}^\co$. However, for any $i \in I$, we have $I-\{i\} \in \mc{F}^\co \subseteq \mc{F}$ but, since $\mc{F}$ is ultrafilter, $\{i\} \in \mc{F}$. However, clearly $\{i\} \in \mc{F}^i$. Therefore, $F \not = F^i$ for any $i \in I$. Likewise we could argue about set of odd numbers.
  \end{proof}

  \begin{exercise}
    An ultrafilter on finite set must be principal.
  \end{exercise}
  \begin{proof}
    Let the set be $A=\{a_1,a_2,\cdots, a_n\}$ and $\mc{F}$ be ultrafilter on it. Since $\{a_1\} \cup \{a_2\}\cup \cdots \cup \{a_n\}=A \in \mc{F}$, it must be the case that $a_i \in \mc{F}$ for some $i$ such that $1 \leq i \leq n$. Since $\mc{F}$ is proper, for any $B \in \mc{F}$, we have $B \cap \{a_i\} \not = \emptyset$, but that means $\mc{F}=\mc{F}^{a_i}$.
  \end{proof}

  \begin{exercise}
    For $\mc{H} \subseteq \ms{P}(I)$. \\
      (i) Show that $\mc{F}^\mc{H}$ is a filter that includes $\mc{H}$, i.e. $\mc{H} \subseteq \mc{F}^\mc{H}$. \\
      (ii) Show that $\mc{F}^\mc{H}$ is included in any other filter that includes $\mc{H}$.
  \end{exercise}
  \begin{proof} $ $ \\
    (i) By the definition.\\
    (ii) Let $\mc{F}$ be a filter such that $H \subseteq \mc{F}$. Let $A \in \mc{F}^\mc{H}$, then $B_1 \cap B_2 \cap \cdots \cap B_n \subseteq A$ for some $B_1,\cdots, B_n \in \mc{H}$. Clearly, $B_1 \cap \cdots \cap B_n \in F$ and so is $A \in F$.
  \end{proof}

  \begin{exercise}
    Let $\mc{F}$ be a proper filter on $I$. \\
    (I) Show that $\mc{F} \cup \{A^c\}$ has fip iff $A \notin \mc{F}$. \\
    (II) Use (I) to deduce that $\mc{F}$ is an ultrafilter iff it is maximal proper filter on $I$.
  \end{exercise}
  \begin{proof} $ $ \\
    (I) First we prove forward implication. Notice that if $A \in \mc{F}$ then since $A \cap A^c=\emptyset$, it follows $\mc{F} \cup \{A^c\}$ does not have fip. \\
    We prove the converse implication. Notice that since $F$ is a proper filter, ie it does not contain $\emptyset$, the only way for $\mc{F} \cup \{A^c\}$ to not have fip would be $A \in \mc{F}$. \\\\

    (II) We prove the foward implication. Suppose $\mc{F}$ is an ultrafilter. Applying Zorn's lemma on proper filter $\mc{F}$, we see that there exists a maximal proper filter, $K$, such that $\mc{F} \subseteq K$. If $K \not = F$, it would mean there is some $A \in K$ such that $A \not \in \mc{F}$ but then that would mean $A^C \in \mc{F}$, and so $A^c \in K$ which contradicts the fact that $K$ is proper since it fails to satisfy fip. Therefore, we must have $F=K$. \\

    We prove the converse implication. Suppose $\mc{F}$ is maximal proper filter. For the sake of contradiction assume $\mc{F}$ is not an ultra filter i.e. there exists some $A \in \ms{P}(I)$ such that both $A, A^c \notin F$. Clearly, either $\mc{F} \cup \{A\}$ has fip or $\mc{F} \cup \{A^c\}$ does. Wlog assume $\mc{F} \cup \{A\}$ has fip. Applying Zorn's lemma on $\mc{F} \cup \{A\}$, we see that there is maximal proper filter $K$ such that $\mc{F} \cup \{A\} \subseteq K$. However this means that $\mc{F}$ is not maximal proper filter since $F \subsetneqq F \cup \{A\} \subset P(I)$.
  \end{proof}

\chapter{Ultrapower Construction of the hyperreals}

  \begin{exercise}
    (section 3.3, I). $\equiv$ is an equivalence relation on $\R^\N$.
  \end{exercise}
  \begin{proof} Let $\langle a_n \rangle, \langle b_n \rangle, \langle c_n \rangle \in \R^\N$.
    First we prove that reflexivity. Clearly, $\{n \in \N: a_n=a_n\}=\N \in \mc{F}^\co \subseteq \mc{F}$. Thus, $\langle a_n \rangle \equiv \langle a_n \rangle$. \\
    Symmetricity is trivial since $\{n \in \N: a_n=b_n\}=\{n \in \N: b_n=a_n\}$, if one of them is in $F$ then so is the other. \\
    Finally, we prove transitivity. Suppose $\la a_n \ra \equiv \la b_n \ra$ and $\la b_n \ra \equiv \la c_n \ra$ i.e. $\{n \in \N: a_n=b_n\}, \{n \in \N: b_n=c_n\} \in \mc{F}$. Clearly, $\{n \in \N: a_n=b_n=c_n\}=\{n \in \N: a_n=b_n\} \cap \{n \in \N: b_n=c_n\} \in \mc{F}$. Therefore, $\la a_n \ra \equiv \la c_n \ra$.
  \end{proof}
  \begin{exercise} (section 3.3, II)
    $\equiv$ is a congruence on the ring $\la \R^\N, \op, \od \ra$, which means that if $r \equiv r^\p$ and $s \equiv s^\p$, then $r \op s=r^\p \op s^\p$ and $r \od s=r^\p \od s^\p$.
  \end{exercise}
  \begin{proof}
    Suppose $r \equiv r^{\p}$ and $s \equiv s^{\p}$ i.e. $ \{n \in \N: r_n=r^\p_n \}, \{n \in \N: s_n=s_n^\p \} \in \mc{F}$. Thus, since $ \{n \in \N: r_n=r_n^\p, s_n=s_n^\p \}=\{n \in \N: r_n=r_n^\p \} \cap \{n \in \N: s_n=s_n^\p \} \in \mc{F}$, we have $\{n \in \N: r_n=r_n^\p, s_n=s_n^\p \} \subseteq \{n \in \N: r_n+s_n=r_n^\p+s_n^\p \} \in \mc{F}$. Similarly, $\{n \in \N: r_ns_n=r_n^\p s_n^\p \} \in \mc{F}$. Therefore, $r\op s \equiv r^\p \op s^\p$ and $r \od s \equiv r^\p \od s^\p$.
  \end{proof}
  \begin{exercise} (section 3.3, III)
    $\la 1, \frac{1}{2}, \frac{1}{3}, \cdots \ra \not \equiv \la 0, 0, 0, \cdots \ra$. 
  \end{exercise}
  \begin{proof}
    Let $\la r \ra=\la 1, \frac{1}{2}, \frac{1}{3}, \cdots \ra$. For the sake of simplicity, define $[[r \not =0]]=\{n \in \N:r_n \not =0\}$. Clearly, $[[r \not =0]] \subseteq \N$. Notice that for any $n \in \N$, we have $n \in [[r \not =0]]$. Thus, $[[r \not =0]]=\N$. So, $[[r \not =0]] \in \mc{F}^\co \subseteq \mc{F}$ but then it would mean $[[r=0]]=[[r \not=0]]^c \notin \mc{F}$. Therefore, $\la r_n \ra \not \equiv \la 0 \ra$.
  \end{proof}

  \begin{exercise}
    (section 3.5, I)
    $[[r=s]] \cap [[s=t]] \subseteq [[r=t]]$.
  \end{exercise}
  \begin{proof}
    Let $n \in [[r=s]] \cap [[s=t]]$. Then it follows that $r_n=s_n$ and $s_n=t_n$, and so $r_n=t_n$. Therefore, $n \in [[s=t]]$.
  \end{proof}

  \begin{exercise} (Section 3.5, II)
    $[[r=r^\p]] \cap [[s=s^\p]] \subseteq [[r \op s=r^\p \op s^\p]] \cap [[r \op s=r^\p \op s^\p]]$
  \end{exercise}
  \begin{proof}
    Let $n \in [[r=r^\p]] \cap [[s=s^\p]]$. Thus, $r_n=r_n^\p$ and $s_n=s_n^\p$, and so $r_n \op s_n=r_n^\p \op s_n^\p$ and $r_n \od s_n = r_n^\p \od s_n^\p$. Therefore, $n \in [[r \op s=r^\p \op s^\p]] \cap [[r \op s=r^\p \op s^\p]]$.
  \end{proof}
  \begin{exercise}
    (Section 3.5, III)
    $[[r=r^\p]] \cap [[s=s^\p]] \cap [[r<s]] \subseteq [[r^\p<s^\p]]$.
  \end{exercise}
  \begin{proof}
    Let $n \in [[r=r^\p]] \cap [[s=s^\p]] \cap [[r<s]]$. Thus, $r_n=r_n^\p, s_n=s_n^\p, r_n<s_n$ but then that means $r_n^\p<s_n^\p$, and so $ n \in [[r^\p<s^\p]]$.
  \end{proof}
  \begin{exercise} (Section 3.5, IV)
    If $r \equiv r^\p$ and $s \equiv s^\p$, then $[[r<s]] \in \mc{F}$ iff $[[r^\p<s^\p]] \in \mc{F}$.
  \end{exercise}
  \begin{proof}
    Suppose $r \equiv r^\p$ and $s \equiv s^\p$, that is to say, $\{n \in \N: r_n=r_n^\p\}, \{n \in \N: s_n=s_n^\p\} \in \mc{F}$. First assume $[[r<s]] \in \mc{F}$. Therefore, $\{n \in \N: r_n=r_n^\p\} \cap \{n \in \N: s_n=s_n^\p\} \cap \{n \in \N: r_n<s_n\} \subseteq \{n \in \N: r^\p<s^\p \} \in mc{F}$. Next assume $[[r^\p<s^p]] \in \mc{F}$. It follows $\{n \in \N: r_n=r_n^\p\} \cap \{n \in \N: s_n=s_n^\p\} \cap \{n \in \N: r_n^\p<s_n^\p\} \subseteq \{n \in \N: r<s\} \in \mc{F}$
  \end{proof}

  \begin{exercise}
    (Section 3.8, I)
    Use only general properties of ordered fields to deduce from the fact that $[\ep]$ is a positive infinitismal, the conclusion that $[\ep]^{-1}$ is greater than every real number.
  \end{exercise}
  \begin{proof}
    Notice that for any $r \in \R^+$, we have ${}^* r=[\la r,r,\cdots,\ra] \in {}^* \R$. Since $\ep<{}^* r$, by fields properties, we have $0<{}^* r^{-1}<\ep^{-1}$. Since $r$ is arbitrary positive real number, setting $r=r^{-1}$,we see that ${}^* r< \ep^{-1}$ for any $r \in \R^+$. 
  \end{proof}
  
  \begin{exercise} (Section 3.10, I)
    If $A$ is finite, show that ${}^*A=A$, and hence $A$ has non nonstandard members.
  \end{exercise}
  \begin{proof}
    Let $A=\{a_1,\cdots, a_n\}$. Consider some $[r] \in {}^* A$. If we could prove that $r \equiv t$ for some constant sequence $t \in A$, we would be done. So we set that as our goal. Notice that since $[r] \in {}^* A$, we have $\{ n \in \N: r_n \in A\}= \{n \in \N: r_n=a_1 \lor r_n=a_2 \lor \cdots r_n=a_n\}=\{n \in \N: r_n=a_1\} \cup \cdots \cup \{n \in \N:r_n=a_n\} \in \mc{F}$. But since $\mc{F}$ is ultrafilter, only one of such set is in $\mc{F}$. Therefore, $\{n \in \N: r_n=a_k\} \in \mc{F}$ for some $k$ such that $1 \leq k \leq n$. Hence, $r \equiv a_k$.
  \end{proof}

  \begin{exercise}
    (Section 3.10, II) 
    Prove: \\
    (a) $A \subseteq B$ iff ${}^* A \subseteq {}^* B$. \\
    (b) $A=B$ iff ${}^* A = {}^* B$.
  \end{exercise}
  \begin{proof} $ $ \\
    (a) First we prove the forward implication. Suppose $A \subseteq B$. Consider $[r] \in {}^* A$. Therefore, we have $[[r \in A]] \in \mc{F}$, but since $[[r \in A]] \subseteq [[r \in B]]$, we have $[[r \in B]]$ and so $[r] \in {}^* B$. \\
    We prove the converse implication. Suppose ${}^* A \subseteq {}^* B$. Consider some $r \in A$. Then we have $[\mb{r}] \in {}^* A$, and so $[\mb{r}] \in {}^*B$. However, since $[{\mb{r}}] \in B$, we have $[[r \in B]] \in \mc{F}$ which means $r \in B$. \\

    (b) Trivially follows from (a).
  \end{proof}

  \begin{exercise}
    (Section 3.10, III) \\
    Prove: \\
    (a) ${}^* (A \cup B) = {}^* A \cup {}^* B$, \\
    (b) ${}^* (A \cap B)= {}^* A \cap {}^* B$, \\
    (c) $\hyp (A-B)=\hyp A - \hyp B$, \\
    (d) $\hyp \emptyset =\emptyset$.
  \end{exercise}
  \begin{proof} $ $ \\
    (a) First we prove the foward implication. Since $A \subset A \cup B$, applying (2), we have ${}^* A \subseteq {}^* (A \cup B)$. Similarly, we have ${}^* B \subseteq {}^* (A \cup B)$, and so ${}^* A \cup {}^* B \subseteq {}^* (A \cup B)$. \\
    We prove the converse implication. Let $[r] \in {}^* A \cup {}^* B$. Then we have following cases; \\
    \textbf{case-1.} $[r] \in {}^* A$. Thus, $[[r \in A]] \in mc{F}$, and since $[[r \in A]] \subseteq [[r \in A \cup B]]$, we have $[[r \in A \cup B]] \in \mc{F}$ but that means $[r] \in {}^* A \cup {}^* B$. \\
    \textbf{case-2.} $[r] \in {}^* B$. The proof is analogous to \textbf{case-1}. \\

    (b) First we prove the forward implication. Since $A \cap B \subseteq A$, applying (2), we have ${}^* (A \cap B) \subseteq {}^* A$. Similarly we have $\hyp (A \cap B) \subseteq \hyp B$, and so $\hyp (A \cap B) \subseteq \hyp A \cap \hyp B$. \\
    We prove the converse implication. Consider some $[r] \in \hyp A \cap \hyp B$. It means that $[r] \in \hyp A$ and $[r] \in \hyp B$, thus $[[r \in A]], [[r \in B]] \in \mc{F}$. Therefore, $[[r \in A \cap B]]=[[r \in A]] \cap [[r \in B]] \in \mc{F}$, and so $[r] \in \hyp (A \cap B)$.  \\

    (c) First we prove the foward implication. Let $[r] \in \hyp (A \setminus B)$. Then we have $[[r \in A]] \cap [[r \notin B]]=[[r \in A \setminus B]] \in \mc{F}$. Thus, $[[r \in A]], [[r \notin B]] \in \mc{F}$. Hence, $[r] \in \hyp A$ and $[r] \notin \hyp B$, and so $[r] \in \hyp A \setminus \hyp B$. \\
    We prove the converse implication. Let $[r] \in \hyp A \setminus \hyp B$ i.e. $[r] \in \hyp A$ and $[r] \notin \hyp B$. Then we have $[[r \in A]] \in \mc{F}$ and $[[r \notin B]] \in \mc{F}$. Thus, $[[r \in A \setminus B]]=[[r \in A]] \cap [[r \notin B]] \in \mc{F}$. Therefore, $r \in \hyp (A \setminus B)$. \\

    (d) Clearly, $\hyp \emptyset =\{ [r] \in \hyp \R: [[r \in \emptyset]] \in \mc{F}\}=\emptyset$. 
  \end{proof}

  \begin{exercise}
    (Section 3.10, IV)
    Is it true that $\hyp (\bigcup_{n=1}^\infty A_n)=\bigcup_{n=1}^\infty \hyp A_n$?
  \end{exercise}
  \begin{proof}
    No. Define $A_n=\{n\}$ for any $n \in \N$. Notice that since $A_n$ is finite, using (1), we have $\hyp A_n=A_n=\{n\}$. Since $[[\omega \in \bigcup_{n=1}^\infty A_n]]=\N \in \mc{F}$, we have $\omega \in \hyp (\bigcup_{n=1}^\infty A_n)$. However, $[[\omega \in A_n]]=\{n\} \notin \mc{F}$, thus $\omega \notin \hyp A_n$ for any $n \in \N$. This mean $\omega \notin \bigcup_{n=1}^\infty \hyp A_n$.
  \end{proof}

  \begin{exercise}
    (Section 3.10, V)
    Show that if $A \subseteq \R$, then $\hyp A \cap \R =A$.
  \end{exercise}
  \begin{proof}
    Suppose $A \subseteq \R$. Let $[r] \in \hyp A \cap \R$. Since $\la r \ra$ is a constant sequence and $[r] \in \hyp A$, it follows $[r]=\la r \ra \in A$. Thus, $\hyp A \cap \R \subseteq A$. This means that $\hyp A \cap \R$ contains only the constant sequences. Let $\la r \ra \in A$. Then clearly $[r] \in \hyp A \cap \R$.
  \end{proof}

  \begin{exercise}
    For $a,b \in \R$, let $[a,b]$ be the closed interval $\{x \in \R: a \leq x \leq n\}$. Prove that $\hyp [a,b]=\{x \in \hyp \R: a \leq x \leq b\}$.
  \end{exercise}
  \begin{proof}
    We first prove the forward inclusion. Let $[r] \in \hyp [a,b]$. Then we have $\{n \in \N: a \leq r_n \leq b\}=[[r \in [a,b]]] \in \mc{F}$ but that means $[r] \in \{x \in \hyp \R: a \leq x \leq b\}$. We prove the converse inclusion. Let $[r] \in \{x \in \hyp \R: a \leq x \leq b\}$. It follows $[[r \in [a,b]]]=\{n \in \N: a \leq r_n \leq b\} \in mc{F}$. Therefore, $[r] \in \{x \in \R: a \leq x \leq n\}$.
  \end{proof}

  \begin{exercise}
    (Section 3.10, VII) 
    $\hyp \Z$ is a subring of $\hyp \R$.
  \end{exercise}
  \begin{proof}
    Clearly, $\hyp \Z \not = \emptyset$. Consider some $[r], [s] \in \hyp \Z$. Then we have $[[r \in \Z]], [[s \in \Z]] \in \mc{F}$. Therefore, since $[[r \in \Z]] \cap [[s \in \Z]] \subseteq [[r+s \in \Z]], [[rs \in \Z]]$, we have $[[r+s \in \Z]], [[rs \in \Z]] \in \mc{F}$. Hence, $[r+s], [rs] \in \hyp \Z$.
  \end{proof}

  \begin{exercise}
    (Section 3.10, VIII)
    If $\R^+=\{x \in \R: x>0\}$, show that $\hyp (\R^+)=\{x \in \hyp \R^+:x>0\}$, i.e., $\hyp (R^+)=(\hyp \R)^+$.
  \end{exercise}
  \begin{proof}
    First we prove the forward inclusion. Let $[r] \in \hyp (R^+)$. It means that $\{n \in \N: r_n>0\}=[[r \in R^+]] \in \mc{F}$ which implies $[r] \in (\hyp \R)^+$. Next we prove the converse inclusion. Let $[r] \in (\hyp \R)^+$. Then we have $\{n \in \N: r_n>0\}=[[r>0]]=[[r \in \R^+]] \in \mc{F}$ from which it follows $[r] \in \hyp(\R^+)$.
  \end{proof}

  \begin{exercise}
    (Section 3.12, I)
    Show that $\hyp f$ agrees with $f$ on $\R$: if $r \in \R$, then $\hyp f(r)=f(r)$.
  \end{exercise}
  \begin{proof}
    Let $r \in \R$. Notice that $r$ can be indetified with the constant sequence $\mb{r} \in \hyp R$. Then we have $\hyp f([\mb{r}])=[f(\mb{r})]$ which is isomorphic to $f(r)$.
  \end{proof}
  \begin{exercise}
    (Section 3.12, II)
    If $f$ is injective, so is $\hyp f$. What about surjectivity?
  \end{exercise}
  \begin{proof}
    Suppose $f$ is injective. Consider some $[r], [s] \in \hyp \R$ such that $\hyp f([r])=\hyp f([s])$, that is to say $[f \circ r]=[f \circ s]$. Since $f$ is injective, we have $[[f \circ r=f \circ s]] \subseteq [[r =s]]$ but then, since $[[f \circ r=f \circ s]] \in \mc{F}$, it follows $[[r=s]] \in \mc{F}$. Therefore, $[r]=[s]$. \\

    Yes, surjecitivity is preserved. Suppose $f$ is surjective. Consider some $[s] \in \hyp \R$. Our goal is to prove that there is some $[r] \in \hyp \R$ such that $\hyp f([r])=[s]$. Define sequenece $(r_n)$ such that, for each $s_n$, $f(r_n)=s_n$. We know such a $r_n$ exists for each $s_n$ since $f$ is surjective. We claim that $[r] \in \hyp \R$ the element we looking for. To prove this claim, we need to show $[f \circ r]=f([r])=[s]$, so we set that as our goal. Clearly, $[[f \circ r = s]]=\{n \in \N: f(r_n)=s_n\}=\N \in \mc{F}$. Therefore, $[f \circ r]=f[r] = [s]$. 
  \end{proof}

  \begin{exercise}
    (Section 3.12, III)
     For $x \in \hyp \R$, let $$|x| = \begin{cases} x, \text{ if } x>0 \\ 0, \text{ if } x=0 \\ -x \text{ if } x<0 \end{cases}$$ be the usal definition of absolute value function. Show that this extends the definition of $|.|$ on $\R$: $|[r]|=[ \la |r_1|, |r_2|,\cdots \ra]=[|r_n|]$.
  \end{exercise}
  \begin{proof}
    We have the following cases; \\
    \textbf{Case-1.} Suppose $[r]>0$. Then $[[r>0]] \in \mc{F}$. Since $[[r>0]] \subseteq [[ [r]=[|r|]]] \in \mc{F}$, we have $|[r]|=[r]=[|r_n|]$. \\
    \textbf{Case-2.} Suppose $[r]=0$. That is to say $[[r=0]] \in \mc{F}$. Since $[[r=0]] \subseteq [[0=|r|]] \in \mc{F}$, we have $|[r]|=0=[|r|]$. \\
    \textbf{Case-3.} Suppose $[r]<0$. Then $| [r]|=-[r]=[-r]=[|r|]$.
  \end{proof}

  \begin{exercise}
    (Section 3.12, IV)
    Let $\chi_A$ be the characteristic function of a set $A \subseteq \R$. Show that $\hyp (\chi_A)=\chi_{\hyp A}$.
  \end{exercise}
  \begin{proof}
    Let $[r] \in \hyp \R$. Consider the expression $\hyp (\chi_A)[r]$, we have the following cases; \\
    \textbf{Case-1.} Suppose $[r] \in \hyp A$. Notice that $\hyp (\chi_A)[r]=[\chi_A \circ r]=[\chi_A(r_n)]$ but then, since $[[r \in A]] \in \mc{F}$ and $[[r \in A]] \subseteq [[\chi_A \circ r=1]]$, we have $[[\chi_A \circ r=1]] \in \mc{F}$. Therefore, $[\chi_A \circ r]=\chi_A [r]=1$. \\
    \textbf{Case-2.} Suppose $[r] \notin \hyp A$. Then we have $[[r \notin A]]=[[r \in A]]^c \in \mc{F}$. Clearly, $[[r \notin A]] \subseteq [[\chi_A \circ r=0]]$. Thus, $[[\chi_A \circ r=0]] \in \mc{F}$ and so $[\chi_A \circ r]=\chi_A [r]=0$.
  \end{proof}
  \begin{exercise}
    (Section 3.12, V)
    Show how to define $\hyp f$ when $f$ is a function of more than one arguement.
  \end{exercise}
  \begin{proof}
    Define $\hyp f([r^1],\cdots,[r^n])=[f(r^1_1, \cdots, r^n_1),f(r^1_2,\cdots, r^n_2), \cdots]$. We prove that it's well defined. Clearly, $[[r^1=r^{1^\p}]] \cap \cdots [[r^n=r^{n^\p}]] \subseteq [[f \circ (r^1,\cdots, r^n)=f \circ (r^{1^\p}, \cdots, r^{n^\p})]]$. Therefore, $r^1 \equiv r^{1^\p}, \cdots, r^n \equiv r^{n^\p}$ implies $f \circ (r^1,\cdots, r^n) \equiv f \circ (r^{1^\p}, \cdots, r^{n^\p}$.
  \end{proof}

  \begin{exercise}
    (Section 3.15, I)
    If $A_1,\cdots, A_k$ are subsets of $\R$, put $P=A_1\times \cdots \times A_k$ and apply the definition of $\hyp P$ to show that $$ \hyp (A_1 \times \cdots \times A_k)=\hyp A_1 \times \cdots \times \hyp A_k.$$
  \end{exercise}
  \begin{proof}
    For $([r^1], [r^2], \cdots, [r^k]) \in \hyp P$, we must have $[[(r^1, r^2, \cdots, r^k) \in P]] \in \mc{F}$. Clearly, $[[(r^1, r^2, \cdots, r^k) \in P]] \subseteq [[r^n \in A_n]]$, thus $[[r^n \in A_n]]$ for any $n$ such that $1 \leq n \leq k$. This means that $[r^n] \in A^n$ for all such $n$. Hence, $([r^1], \cdots, [r^n]) \in \hyp A_1 \times \cdots \hyp A_n$. From this, it follows that $\hyp (\R^k)=(\hyp \R)^k$.
  \end{proof}

  \begin{exercise}
    (Section 3.15, II)
    Let $\dom P$ denote the domain of a binary relation. If $P \subseteq \R^2$, show that $\hyp (\dom P)= \dom \hyp P$.
  \end{exercise}
  \begin{proof}
    First we prove the forward inclusion. Let $[r] \in \hyp (\dom P)$. Then we have $\{n \in \N: r_n \in \dom P\}=[[r \in \dom P]] \in \mc{F}$. $$\text{ Define sequence } s_n = \begin{cases} s \text{ such that } (r_n, s) \in P, \text{ if } r_n \in \dom P \\ 0, \text{ if } r_n \notin \dom P \end{cases}$$ Then clearly, $[[(r,s) \in P]]=[[r \in \dom P]] \in \mc{F}$, but that means $([r], [s]) \in \hyp P$, and so $[r] \in \dom \hyp P$. \\
    Now we prove the converse inclusion. Let $[r] \in \dom \hyp P$. It follows that there exists $[s] \in \hyp \R$ such that $([r], [s]) \in \hyp P$, but that means $[[r \in \dom P]]=[[(r,s) \in P]] \in \mc{F}$. Hence, $[r] \in \hyp (\dom P)$.
  \end{proof}

\chapter{The Transfer Principle}

  \begin{exercise}
    (Section 4.1, I)
    Verify the truth the $*-$transform given in $1-5$ above, which are; \\
    (1) The Eudoxus-Archimedes Principle: $\forall x \in \hyp \R: \exists m \in \hyp \N (x<m)$.  \\
    (2) Density of the rationals: $\forall x, y \in \hyp \R (x<y \text{ implies } \exists q \in \hyp \Q (x<q<y))$. \\
    (3) Finiteness. Let $A=\{r_1,\cdots, r_k\}$ be a finite subset of $\R$. Then $\forall x \in \hyp A (x=\hyp r_1 \lor x=\hyp r_2 \lor \cdots x=\hyp r_k)$. \\
    (4) Finitary set operations. $\forall x \in \hyp \R (x \in \hyp (A \cup B))$ iff $x \in \hyp A$ or $x \in \hyp B$. \\
    (5) Discreteness of $\N$.  $\forall x \in \hyp \N (\hyp n \leq x \leq \hyp (n+1) \text{ implies } x= \hyp n \text{ or } x=\hyp (n+1)$.
  \end{exercise}
  \begin{proof} $ $ \\

    (1) Let $x \in \hyp \R$. Define sequence $m$ such that $x_i<m_i$ for every $i$. The existence of such $m$ is guranteed by Eudoxus-Archimedes Principle in reals. Clearly, $m \in \hyp \N$ and $[[x<m]] \in \mc{F}$. Therefore, $x<m$. 

    (2) Let $x,y \in \hyp \R$. Define sequence $q$ such that $x_i<q_i<y_i$ for each $i$, where $q_i$'s are rational. The existence of such $q_i$'s is guranteed by density of rationals in reals. Clearly, $q \in \hyp \Q$ and $[[x<q<y]]=\N \in \mc{F}$. Therefore, $x<q<y$. 

    (3) Let $x \in \hyp A$. Then $[[x=r_1]] \cup [[x=r_2]]\cup \cdots [[x=r_n]]=[[x \in A]] \in \mc{F}$. Since finite union in $\mc{F}$ means that exactly one of the set in $\mc{F}$, we have $x=\hyp r_1$ or $x=\hyp r_2$ or $\cdots$ or $x=\hyp r_n$.

    (4) Proved in section 3.10, II.

    (5) Let $x \in \hyp \N$ and suppose $\hyp n \leq x \leq \hyp (n+1)$. Then we have $[[n \leq x \leq n+1]] \in \mc{F}$ but due to discreteness of $\N$, it folows that $[[x=n]] \cup [[x=n+1]]=[[x=n \lor x=n+1]]=[[n \leq x \leq n+1]] \in \mc{F}$. Hence, $x=\hyp n$ or $x=\hyp(n+1)$.
  \end{proof}

\chapter{Hypereals Great and Small}

  \begin{exercise}
    (Section  5.2, I)
    For any positive hyperreal $a$, explain why the function $x \rightarrow a^x$ is defined for all $x \in \hyp \R$. Use transfer to explore its properties.
  \end{exercise}
  \begin{proof}
    
  \end{proof}

  \begin{exercise}
    (Section 5.5, I)
    Verify that $\simeq$ and $\sim$ are equivalence relationships.
  \end{exercise}
  \begin{proof}
    We will first argue about $\simeq$. Let $x,y,z \in \hyp \R$. Clearly, it's reflexive since $x-x=0$. If $x \simeq y$ then we have $x-y= \epsilon$, thus $y-x=-\epsilon$, and so $y \simeq x$. If $x \simeq y$ and $y \simeq z$ then we have $x-y=\epsilon$ and $y-z=\delta$, adding the two get us $x \simeq z$. The arguement of $\sim$ is similar, so we omit the details.
  \end{proof}

  \begin{exercise}
    (Section 5.5, II)
    If $b \simeq x \leq y \simeq c$ with $b$ and $c$ real, show that $b \leq c$. What if $b$ and/or $c$ are not real?
  \end{exercise}
  \begin{proof}
    First we will prove an useful lemma, if $y>0$ and $y \simeq k$ then $k>0$. Let $k=y+\ep$. In the case where $\ep=0$ and $\ep>0$, it follows trivially. So assume $\ep<0$. Thus, $y-\frac{1}{n}<y+\ep=k$ for all $n \in \N$. Since $\frac{y}{2}$ a real number, we know that there exists $m \in \N$ such that $\frac{1}{m}<\frac{y}{2}$ which implies $0<\frac{y}{2}=y-\frac{y}{2}<y-\frac{1}{m}<k$. This proves the lemma. \\
    Now assume to the contrary that $c<b$. Thus, $0<c-b$ but it implies that $0<x-y$, since $c-b \simeq x-y$, which is a contradiction.\\
    The result would not be true if $b$ is not real. Consider $\ep \simeq 0 \leq 0 \simeq 0$ for some positive infintisimal $\ep$, this would imply $\ep \leq 0$ which is false. 
  \end{proof}

  \begin{exercise}
    (Section 5.5, III)
    $\hal(b)=\{b+\ep: e \in \hal (0)\}$.
  \end{exercise}
  \begin{proof}
    The converse inclusion is trivial. For forward, notice that $x \in \hal (b)$ means that $x-b=\ep$, and so $x=b+\ep$.
  \end{proof}

  \begin{exercise}
    (Section 5.5, IV)
    $\gal(b)=\{b+c: c\in \gal (0) \}$.
  \end{exercise}
  \begin{proof}
    Again, the converse implication is trivial. For forward, we see that $x \in \gal(b)$ means $x-b=c$ where $c$ is a limited number, and so $c \in \gal(0)$. Thus, $x=b+c$ for some $c \in \gal(0)$.
  \end{proof}
  
  \begin{exercise}
    (Section 5.5, V)
    If $x \simeq y$ and $b$ is limited, prove that $b \cdot x \simeq b \cdot y$. Show that the result can fail for unlimited $b$. 
  \end{exercise}
  \begin{proof}
    Since $x \simeq y$, we have $x-y=\ep$ for some $\ep \in \hal(0)$. Multiplying by $b$ on both sides get us, $bx-by=b \cdot \ep$. It follows that $b \cdot \ep \in \hal(0)$, so $bx \simeq by$. Consider $b=\frac{1}{\ep}$then $b \cdot \ep=1$ which is not infinitisimal.
  \end{proof}

  \begin{exercise}
    (Section 5.5, VI)
    Show that any galaxy contains members of $\hyp \Z$, of $\hyp \Q - \hyp \Z$, and of $\hyp \R - \hyp \Q$.
  \end{exercise}
  \begin{proof}
    Let $b \in \hyp \R$. Clearly, $b-a$ for $a \in \hyp Z$, such that $a$ is intger, is limited. 
  \end{proof}

  \begin{exercise}
    (Section 5.7, I)
    Prove theorem 5.6.2: \\
    (a) $\sh (b\pm c) = \sh (b) \pm \sh (c)$. \\
    (b) $\sh(bc)=\sh(b)\cdot \sh(c)$. \\
    (c) $\sh(b/c)=\sh(b)/\sh(c)$ if $\sh(c) \not = 0$ (ie $c$ is appreciable).\\ 
    (d) $\sh(b^n)=\sh(b)^n$. \\
    (e) $\sh (|b|)=|\sh (b)|$.\\
    (f) $\sh(\sqrt[n]{b})=\sqrt[n]{\sh(b)}$, if $b \geq 0$. \\
    (g) If $b \leq c$ then $\sh(b) \leq \sh(c)$.
  \end{exercise}
  \begin{proof} Consider $b=\sh(b)+\ep$ and $c=\sh(c)+\del$ for some infinitisimal $\ep$ and $\del$. \\
    (a) For $\sh(b+c)$, notice that $b+c-(\sh(b)+\sh(c))=\ep+\del$. Thus, $\sh(b+c) \simeq b+c \equiv \sh(b)+sh(c)$. From the uniqueness of shadow, it follows $\sh(b+c)=\sh(b)+\sh(c)$. We get identical result in case of $\sh(b-c)$ by considering $b-c-(\sh(b)-\sh(c))=\ep-\del$. \\

    (b) Notice that $bc=\sh(b)\sh(c)+\sh(b) \del +\sh(c)\ep +\ep\del$ where $\sh(b) \del, \sh(c)\ep, \ep\del$ are infintisimals. So, $\sh(bc) \simeq bc \simeq \sh(b)\sh(c)$, and thus by uniqueness of shadow we have $\sh(bc)=\sh(b)\sh(c)$.\\

    (c) Notice that, applying (b), we have $\sh(b/c)=\sh(b) \cdot \sh(1/c)$. So if we could prove $\sh(1/c)=1/\sh(c)$, we would be done. Therefore, we set that as our goal. Clearly, $\frac{1}{c}-\frac{1}{\sh(c)}=\frac{1}{\sh(c)+\del}-\frac{1}{\sh(c)}=\frac{\del}{\sh(c)^2+\sh(c)\del}$. Since $\sh(c)^2+\sh(c)\del$ is appreciable it follows that $\frac{\del}{\sh(c)^2+\sh(c)\del}$ is infinitismal, and thus $\sh(1/c) \simeq 1/c \simeq 1/\sh(c)$. Hence, $\sh(1/c)=1/sh(c)$ by uniqueness of shadow. \\

    (d) Trivially follows from induction on (b). \\

    (e) If we could prove that $|b|\simeq |\sh (b)|$, we would be done since from uniqueness of shadow, $\sh |b|=|\sh (b)|$ would hold. Notice that $|b|-|\sh (b)| \leq |\sh (b)|+|e|-|\sh (b)|=|e|$. If $0 \leq |b|-|\sh (b)|$ then clearly $|b|-|\sh b| \leq |e|<1/n$ for all $n \in \N$. If $|b|-|\sh (b)|<0$ then $-1/n<-|e| \leq |b|-|\sh(b)|$ for all $n \in \N$. Thus, $||b|-|\sh (b)||<1/n$ for every natural $n$. Hence, $|b|-|\sh(b)$ is infinitisimal as desired. 

    (f) If $\sh(b^\frac{1}{n})<(\sh (b))^\frac{1}{n}$ then by squaring both side and applying part (d), we would get $\sh (b)=\sh(b^{1/n \cdot n})=(\sh (b^{1/n}))^n<(\sh b)^{1/n \cdot n}=\sh(b)$ which is a contradiction. We would get identical contradiction in case of $(\sh (b))^\frac{1}{n}<\sh(b^\frac{1}{n})$. 

    (g) It follows from 2nd exercise of section 5.5.
  \end{proof}

  \begin{exercise}
    (Section 5.7, I)
    Show that if $b,c$ are limited, and $b \simeq b^\p, c \simeq c^\p$, then $b \pm c \simeq b^\p \pm c^\p, b\cdot c \simeq b^\p \cdot c^\p$, and $b/c \simeq b^\p/c^\p$ if $c \not \simeq 0$. Show that the last result can fail when $c \simeq 0$.
  \end{exercise}
  \begin{proof}
    Let $b=b^\p+\ep$ and $c=c^\p +\del$ for some infintisimal $\ep$ and $\del$. Then $b \pm c - (b^\p \pm c^\p)=\ep \pm \del$ which is known to be infinitisimal. Thus, $b \pm c \simeq b^\p \pm c^\p$. For multiplication we have $bc-b^\p c^\p=b^\p c^\p+b^\p \del +c^\p \ep + \ep \del -b^\p c^\p=b^\p \del +c^\p \ep + \ep \del$ which is infintisimal. Finally for quotient, we have $\frac{b}{c}-\frac{b^\p}{c^\p}=\frac{b^\p +\ep}{c^\p+\del}-\frac{b^\p}{c^\p}=\frac{b^\p c^\p+c^\p \ep -b^\p c^\p-b^\p \del}{(c^\p)^2+c^\p \del}=\frac{c^\p \ep+b^\p \del}{(c^\p)^2+c^\p \del}$. Since $(c^\p)^2+c^\p \del$ is appreciable, we have $\frac{b}{c}-\frac{b^\p}{c^\p}$ to be infinitsimal. Hence, $b/c \simeq b^\p/c^\p$. For proving the result does not hold for $c \simeq 0$. Consider $b=0$. Then we have $b/c-b^\p/c^p=0-\frac{\ep}{\epsilon-\delta}$ which is not infinitisimal necessarily.
  \end{proof}

  \begin{exercise}
    (Section 5.7, II) \\
    If $\ep$ is infinitesimal, show that \\
    (a) $\sin \ep \simeq 0$, \\
    (b) $\cos \ep \simeq 1$, \\
    (c) $\tan \ep \simeq 0$, \\
    (d) $\sin \ep /\ep \simeq 1$, \\
    (e) $(\sin \ep)/\ep \simeq 1$,\\
    (f) $(\cos \ep -1)/\ep \simeq 0$.
  \end{exercise}
  \begin{proof} $ $ \\
    (a) Since $\forall x \in \R (|\sin x| \leq |x|)$ holds in $\R$, applying transfer on it, it must be the case that $\forall x \in \hyp \R (|\sin x| \leq |x|)$ holds in $\hyp \R$. Therefore, $|\sin \ep| \leq |\ep|$, and so $\sin \ep$ is infintisimal. Hence, $\sin \ep \simeq 0$. \\

    (b) Since $\cos \ep=\sqrt{(1-\sin^2 \ep)}$, we have $\sh (cos \ep)=\sh \sqrt{(1-\sin^2 \ep)}=\sqrt{(\sh(1)-\sh(\sin^2 \ep))}=\sqrt{(1-\sh(\del^2))}=\sqrt{(1-0)}=1$. Clearly, $\cos \ep \simeq \sh(\cos \ep)=1$. \\

    (c) $\sh (\tan \ep)=\sh(\sin \ep)/\sh( \cos \ep)=0/1=0$. Hence, $\tan \ep \simeq 0$. \\

    (d) Notice that $|\sin \ep| \leq |\ep|$, so $|\sin (\ep)/\ep| \leq 1$. Thus, $\sh \left( \left| \frac{\sin \ep}{\ep} \right| \right) \leq 1$. We will now show that $\sh \left( \left| \frac{\sin \ep}{\ep} \right| \right)<1$ leads to contradiction which would get us $\sh \left( \left| \frac{\sin \ep}{\ep} \right| \right)=1$. Clearly if $\sh \left( \left| \frac{\sin \ep}{\ep} \right| \right)<1$ then $0=\sh(|\sin \ep|)-\sh|\ep|<0$ which is false. Therefore, $(\sin \ep)/\ep \simeq \sh \left( \left| \frac{\sin \ep}{\ep} \right| \right)=1$. \\

    (e) By the sine inequality, $|\sin \ep|/|\ep| \leq 1$. By the tangent inequality, $|x| \leq |\tan (x)|$ for $x \in [0, \pi/2)$, we have $|\cos x|\leq |\sin x|/|x|$. Combining the two inequalities and taking shadow we see $1=\sh (|\cos \ep|) \leq \sh (|\sin(\ep)/\ep |) \leq 1$. Therefore, $\sh(|\sin(\ep)/\ep |)=1$. Hence, $(\sin \ep)/\ep \simeq 1$. \\

    (f) By cosine inequality, $1-\frac{\ep^2}{2} \leq \cos \ep$, we have $-\ep/2 \leq (\cos \ep -1)/\ep$. Since $cos \ep \leq 1$, we have $(\cos \ep -1)/\ep \leq 0/\ep=0$. Hence $|(\cos \ep-1)/\ep| \leq \ep/2$ and so $(\cos \ep-1)/\ep$ is infinitesimal, which makes it infinitely close to $0$.
  \end{proof}

  \begin{exercise}
    (Section 5.7, III)
    Show that every hyperreal is infinitely close to some hyperrational number.
  \end{exercise}
  \begin{proof}
    Since $\forall x,y \in \R (x<y \text{ implies } \exists q \in \Q (x<q<y))$ holds in reals (density of rationals), by transfer, we have $\forall x,y \in \hyp \R (x<y \text{ implies } \exists q \in \hyp \Q (x<q<y))$. In particular there is a hyperrational $q$ such that $r<q<r+\ep$ for any hyperreal $r$. Clearly such a $q$ is infinitely close to $r$.
  \end{proof}

  \begin{exercise}
    (Section 5.7, IV)
    Show that $\R$ is isomorphic to the ring of limited hyperrationals $\hyp \Q \cap \mathbb{L}$ factored by its ideal $\hyp \Q \cap \I$ of hyperreational infintisimal.
  \end{exercise}
  \begin{proof}
    \[\begin{tikzcd}
      {\bullet {}^*\mathbb{Q} \cap \mathbb{L}} &&&& {\bullet \mathbb{R}} \\
      \\
      && {\bullet ({}^* \mathbb{Q} \cap \mathbb{L})/({}^* \mathbb{Q} \cap \mathbb{I})}
      \arrow["{\text{canonical projection}}"', two heads, from=1-1, to=3-3]
      \arrow["{f(\text{hal}(b)+({}^* \mathbb{Q} \cap \mathbb{L})/({}^* \mathbb{Q} \cap \mathbb{I}))=\text{sh}(b)}"{description}, tail, two heads, from=3-3, to=1-5]
      \arrow["{\text{sh}}"', two heads, from=1-1, to=1-5]
    \end{tikzcd}\]
    Consider shadow function restricted to $\hyp \Q \cap \mathbb{L}$, in what follows we would use $\sh$ notation for it instead of $\sh_{\hyp \Q \cap \mathbb{L}}$. We will first justify it's surjective. Since $\R \subseteq \hyp \R$, applying previous exercise, we see that for any $x \in \R$, there is $q \in \hyp \Q$ such that $x \simeq q$ but then it would mean $x \simeq \sh(q)$. Since both $x$ and $\sh(q)$ are reals, it must be the case that $x=\sh(q)$. Hence, $\sh(q)=x$ as desired. Notice that for any $[r] \in (\hyp \Q \cap \mathbb{L})/(\hyp Q \cap \I)$, we have $[r]=r+\hyp Q \cap \I$ which is nothing but $\hal(r)$. Define $f: (\hyp \Q \cap \mathbb{L})/(\hyp Q \cap \I) \rightarrow \R$ as $\hal(b) \mapsto \sh(b)$. By the first isomorphism theorem of rings, it follows $(\hyp \Q \cap \mathbb{L})/(\hyp Q \cap \I) \simeq \R$.
  \end{proof}

  \begin{exercise}
    (Section 5.9, Exercise on Dedekind completeness)
    For threoem 5.8.1, instead of showing that cauchy sequences converge we can develop a direct proof that any subset $A \subseteq \R$ with a real upper bound has a least real upper bound. First, for each $n \in \N$, let $s_n$ be the least $k \in \Z$ such that $k/n$ is an upper bound of $A$. Then take an unlimited $N \in \hyp \N_\infty$ and let $L \in \R$ be infinitely close to $s_N/N$. \\
    (a) Verify that $s_n$ exists as defined for $n \in \N$. \\
    (b) Show the $s_N/N$ is limited, so that such a real $L$ exists under hypothesis of theorem 5.8.1. \\
    (c) Prove that $L$ is a least upper bound of $A$ in $\R$.
  \end{exercise}
  \begin{proof} Let $r$ be an upper bound of $A \subseteq \R$. \\
    (a) Let $n \in \N$. Consider the set $B=\{m \in \N: m/n \text{ is an upper bound of } A \}$. Since $r \in \R$, by archimedean property, we know that there exists $l \in \N$ such that $r<l$. Clearly $l\cdot n \in B$, and so $B$ is non empty. By the well ordering principle, it follows that there exists a least element in $B$, or in other words there exists a least $k \in \N$ such that $k/n$ is an upperbound of $A$ for each $n$. \\

    (b) Let $n \in \N$. Since $s_n$ is the least integer such that $s_n/n$ is upperbound of $A$, clearly $(s_n-1)/n$ is not an upper bound. But that means there exists some $a \in A$ such that $(s_n-1)/n \leq a \leq s_n/n$. Applying transfer on the statement $(\forall n \in \N) (\exists a \in A) ((s_n-1)/n \leq a \leq s_n/n)$, we get $(\forall n \in \hyp \N) (\exists a \in \hyp A) ((s_n-1)/n \leq a \leq s_n/n)$. In particular, we have $(s_N-1)/N \leq a \leq s_N/N$. However since $s_N/N$ and $(s_N-1)/N$ are infinitely close, it must be the case that $a$ is infinitely close to $s_N/N$. Applying transfer on the statement $(\forall a \in A) (a \leq r)$, where $r$ is the real upper bound we are given, we get $(\forall a \in \hyp A)(a \leq r)$.  Thus, $s_N/N \leq r$. Next we will prove that is not negative unlimited either. Since $A \subseteq \hyp A$ and $s_N/N$ is an upper bound of $\hyp A$, we have some $x \in A$ such that $x \leq s_N/N$. Hence, $x \leq s_N/N \leq r$ for some real $r,s$. \\

    (c) For the sake of contradiction assume there is an upper bound, $K$, of $A$ such that $K<L$. Notice that there does not exists any $a \in A$ such that $a \in [K, L]$ since otherwise $K$ would not be an upper bound. Applying transfer on $(\forall a \in A) (a \notin [K, L])$, we get $(\forall a \in \hyp A)(a \notin \hyp [K, L])$ which is false since in previous paragraph we demonstrated that there is an element of $\hyp A$ which is infinitely close to $L$.
  \end{proof}

  \begin{exercise}
    (Section 5.11, I)
    Provide an analogous description of the order structure of the hyperintgers $\hyp \Z$.
  \end{exercise}
  \begin{proof}
    By transfer, we know that $\hyp \Z$ is closed under addition, subtraction and multiplication. Observe that only limited hyperinteger are members of $\Z$ since if $k \in \hyp Z$ is limited then $|k| leq n$ for some $n \in \N$ but that means $-n \leq k \leq n$. Applying transfer on $\forall x \in \Z (-n \leq x \leq n \rightarrow x=-n \lor x=-n+1 \lor \cdots x=1 \lor \cdots x=n)$, thus $k \in \{-n, -n+1, \cdots, n\}$, and so $k \in \Z$. Therefore, all numbers of $\hyp \Z-\Z$ are unlimited and thus either less than all members of $\Z$ or greater than it. Let $K \in \hyp \Z - \Z$. Define $\gamma (K)=\{K\} \cup \{K \pm n: n \in \N\}$, clearly all members of $\gamma(K)$ are unlimited and $\gamma(K)$ is isomorphic to $\Z$ under the usual ordering. Alternatively, we can defined $\gamma(K)=\{H \in \hyp \Z: Z \sim H\}=\gal(K) \cap \hyp \Z$. Notice that $Z$ itself is a $\hyp \Z$ galaxy, $\Z=\gal(0) \cap \hyp \Z$. \\

    We define order on $\hyp \Z$ galaxy as $\gamma(Y)< \gamma (H)$ iff $Y \leq H$ whenever $K \not \sim H$. Clearly there is no greatest $\hyp \Z$ galaxy since $\gamma(K)< \gamma (2K)$ for any positive unlimited $K$. Simillarly there is no smallest galaxy either. To see why notice that either $K$ is even or $K+1$ is, by transfer. If $K+1$ is even, we have $\gamma(K+1/2)<\gamma(K+1)=\gamma(K)$. If $K$ is even then we have $\gamma(K)<\gamma(K)$. \\

    We can imagine ordering on $\hyp \Z$ as ordering of $\Z$ surrounded by ordered set of $\hyp \Z$ galaxies with no greatest or least.
  \end{proof}

  \begin{exercise}
    (Section 5.11, II)
    Show that for any $M \in \hyp \N$ there is an $N \in \hyp \N$ that is divisble by all members of $\{1,2,\cdots, M\}$. Hence show that there exists a hypernatural number $N$ that is divisble by every standard positive integer.
  \end{exercise}
  \begin{proof}
    We extend the factorial function $!:\N \rightarrow \N$ to $!:\hyp \N \arr \hyp \N$ using extension of function. Applying transfer on the statement $(\forall n \in \N)(k \leq n \arr k|n!)$, we get $(\forall n \in \hyp \N)(k \leq n \arr k|n!)$. Thus, letting $N=M!$ does the job.
  \end{proof}

  \begin{exercise}
    (Section 5.11, III)
    Develop a theory of prime factors in $\hyp \N$: if $\Pi$ is the set of standard prime numbers, with enlargement $\hyp \Pi \subseteq \hyp \N$, prove the following. \\
    (a) $\hyp \Pi$ consists precisely of those hypernaturals $>1$ that have no nontrivial factors in $\hyp \N$. \\
    (b) Every hypernatural number $>1$ has a hyperprime factor, i.e., is divisble by some members of $\hyp \Pi$. \\
    (c) Two hypernaturals are equal if they have exactly the same factors of the form $p^n$ with $p \in \hyp \Pi$ and $n \in \hyp \N$. \\
    (d) A hypernatural number is divisible by every standard positive integer iff it is divisible by $p^n$ for every standard prime $p$ and every $n \in \N$.
  \end{exercise}
  \begin{proof} $ $ \\
    (a) By transfer on $(\forall x \in \Pi) (x>1 \rightarrow \text{ x have no nontrivial factors in } \N)$. \\
    (b) By transfer on $(\forall n \in \N)(n>1 \rightarrow \exists p \in \Pi)(p|n)$.  \\
    (c) By transfer on $(\forall x \in \N)(\forall y \in \N)(x=y \leftrightarrow$ they have exactly the same factors of the form $p^n$ with $p \in \Pi$ and $n \in \N$). \\
    (d) Trivial.
  \end{proof}

\chapter{Convergence of Sequences and Series}

  \begin{exercise}
    (Section 6.3, I)
    Prove theorem 6.3.1. If $\lim_{n \arr \infty }s_n=L$ and $\lim_{n \arr \infty} t_n=M$ in $\R$. Then \\
    (a) $\lim_{n\arr \infty} (s_n+t_n)=L+M$. \\
    (b) $\lim_{n \arr \infty} (cs_n)=cL$, for any $c \in \R$, \\
    (c) $\lim_{n \arr \infty} (s_n t_n)=LM$, \\
    (d) $\lim_{n \arr \infty} (s_n/t_n)=L/M$, if $M \not = 0$.
  \end{exercise}
  \begin{proof} For what follows consider $N \in \hyp \N_{\infty}$. It means that $s_N \simeq L$ and $t_N \simeq M$.\\
    (a) Since $s_N \simeq L$ and $t_N \simeq M$, it follows $s_N+t_N \simeq L+M$. Since $N$ was arbitrary, we have $s_n+t_n \simeq L+M$ for all unlimited $n$. Hence, $\lim_{n \arr \infty } (s_n+t_n)=L+M$. \\
    (b) Since $S_N \simeq L$, we have $c S_N \simeq cL$. Since $N$ was arbitrary, $cs_n \simeq cL$ for all unlimited $n$. Hence, $\lim_{n \arr \infty } cs_n=cL$. \\
    (c) Since $s_N \simeq L$ and $t_N \simeq M$, we have $s_N t_N \simeq LM$ but since $N$ was arbitrary, it follows $s_nt_n \simeq LM$ for all unlimited $n$. Hence, $\lim_{n \arr \infty} (s_n t_n)=LM$. \\
    (d) Again, from $s_N \simeq L$ and $t_N \simeq M$, we have $s_N/t_N \simeq L/M$ but since $N$ was arbitrary, $\lim_{n \arr \infty } s_n/t_n=L/M$.
  \end{proof}
  \begin{exercise}
    (Section 6.4, I)
    Prove theorem 6.4.2. A real valued sequence \\
    (a) diverges to infinity iff all of its extended terms are positive unlimited; and \\
    (b) diverges to negative infinity iff all of its extended terms are negative unlimited.
  \end{exercise}
  \begin{proof} Let the sequence be $(s_n)$. \\
    (a) First we prove forward direction. Let $r \in \R$. Since the sequence diverges, we know that there is some $n \in \N$ such that $(\forall m \in \N)(m \geq n \arr s_m>r)$ holds true. Applying transfer on it, we have $(\forall m \in \hyp \N)(m \geq n \arr s_m>r)$ to be true. In particular, $s_N>r$ for any $N \in \hyp \N_\infty$ since $N>n$. It follows that $s_N$ is positive unlimited for all $N \in \hyp \N_\infty$. We prove the converse direction. Let $r \in \R$. Notice that for any unlimited $N \in \hyp \N_\infty$ if $M\geq N$ then $M \in \hyp \N_\infty$. Therefore, it follows that $(\exists n \in \hyp \N)(\forall m \in \hyp \N)(m \geq n \arr s_m>r)$ since $s_m$ is positive unlimited by assumption. Transferring the statement to $\R$, we have $(\exists n \in \N)(\forall m \in \N)(m \geq n \arr s_m>r)$ to be true in $\R$. Since $r$ was arbitrary, we have $(s_n)$ to be divergent to positive infinity. \\

    (b) Follows from (a) by considering the sequence $(-s_n)$.
  \end{proof}
\newpage
  \begin{exercise}
    (Section 6.5, I)
    Prove theorem 6.5.1. A real values sequence $\la s_n \ra$ is Cauchy in $\R$ iff all it's extended terms are infinitely close to each other, i.e., iff $s_m \simeq s_n$ for all $m,n \in \hyp \N_\infty$.
  \end{exercise}
  \begin{proof}
    First we prove the forward implication. Suppose $\la s_n \ra$ is cauchy. Let $\ep \in \R^+$ and $j \in \N$ be such that $(\forall m,n \in \N)(m,n \geq j \rightarrow |s_m-s_n|<\ep)$. Applying transfer on the latter we know $(\forall m,n \in \hyp \N)(m,n \geq j \rightarrow |s_m-s_n|<\ep)$ is true in $\hyp \R$. In particular, we have $|s_M-s_N|<\ep$ for arbitrary $M, N \in \hyp \N_\infty$ since $j$ is limited. Since $\ep$ was arbitrary positive real number, we have $s_M \simeq s_N$. We now prove the converse direction. Suppose $s_m \simeq s_n$ for all $m,n \in \hyp \N_\infty$. Let $\ep \in \R^+$ be arbitrary. Notice that for any $K \in \hyp \N_\infty$, if $L \geq N$ then $L \in \hyp \N_\infty$, and also notice for any $L, K \in \hyp \N_\infty$, $s_L-s_K=\delta$ where $\delta$ is infintisimal. Therefore, we have $(\exists j \in \hyp \N)(\forall m,n \in \hyp \N)(m,n \geq j \rightarrow |s_m-s_n| \leq \ep)$ to be true in $\hyp \R$. Applying transfer to it, we have $(\exists j \in \N)(\forall m,n \in \N)(m,n \geq j \rightarrow |s_m-s_n|<\ep)$ to be true in $\R$. Since $\ep \in \R^+$ was arbitrary, we have $(s_n)$ to be cauchy. 
  \end{proof}
  
  \begin{exercise}
    (Section 6.5, II). Cauchy's convergence Criterion. \\
    Prove the foward direction of \textit{A real valued sequence converges in $\R$ iff it is cauchy}.
  \end{exercise}
  \begin{proof}
    Let $(s_n)$ be a real valued sequence and suppose it convergence in $\R$ to $L$. Therefore, by theorem 6.1.1, we have $s_N \simeq L$ for all $n \in \hyp \N_\infty$ but that means all extended terms are infinitely close to each other i.e. for any $n,m \in \hyp \N_\infty$, $s_n \simeq L \simeq s_m$. Therefore, by theorem 6.5.1, we have $(s_n)$ to be cauchy.
  \end{proof}

  \begin{exercise}
    (Section 6.7, I)
    Let $(s_n)$ and $(t_n)$ be real valued sequences with limits, $L, M$ respectively. Show that if $s_n \leq t_n$ for $n \in \N$. Then $L \leq M$.
  \end{exercise}
  \begin{proof}
    Applying transfer on $(\forall n \in \N)(s_n \leq t_n)$, we have $(\forall n \in \N)(s_n \leq t_n)$ to be true in $\hyp \R$. In particular for any $N \in \hyp \N_\infty$, we have $s_N \leq t_N$, however, since $L \simeq s_N \leq t_N \simeq M$, we have $L \leq M$ by exercise 2 of section 5.5.
  \end{proof}
  
  \begin{exercise}
    (Section 6.7, II)
    If $r_n \leq s_n \leq t_n$ in $\R$ for all $n \in \N$, and $\lim_{n \arr \infty} r_n=\lim_{n \arr \infty} t_n$, show that $(s_n)$ converges to this same limit.
  \end{exercise}
  \begin{proof}
    Applying transfer on $(\forall n \in \N)(r_n \leq s_n \leq t_n)$, we have $(\forall n \in \hyp \N)(r_n \leq s_n \leq t_n)$ to be true in $\hyp \R$. In particular, for any $N \in \hyp \N_\infty$, we have $r_N \leq s_N \leq t_N$. However, since $(r_N), (t_N)$ are limited, $s_N$ is limited. By the shadow inequalitiy, we have $\sh(r_N) \leq \sh(s_N) \leq \sh(t_N)=\sh(r_N)$. Hence, $\sh(s_N)=\sh(r_N)=\lim_{n \arr \infty} r_n$ but that means $r_N \simeq \lim_{n \arr \infty} r_n$ for all $N \in \hyp \N_\infty$. Hence, $(s_n)$ converges to $\lim_{n \arr \infty} r_n$.
  \end{proof}

  \begin{exercise}
    (Section 6.7, III)
    If a sequence converges in $\R$, show that it has exactly one cluster point.
  \end{exercise}
  \begin{proof}
    Suppose $(s_n)$ converges to $L$ and assume it has two cluster point $A$ and $B$. By definition of convergence we have $s_N \simeq L$ for all $N \in \hyp N_\infty$ but that would imply $B \simeq L \simeq A$ by the definition of cluster point, since they are reals, it follows $A=L=B$. 
  \end{proof}

  \begin{exercise}
    (Section 6.7, IV)
    Suppose that a real valued sequence has a single cluster point. If the sequence is bounded, must it be convergent? what if it is unbounded?
  \end{exercise}
  \begin{proof}
    Let $(s_n)$ be a bounded sequence which has an unique cluster point i.e. for some $M \in \hyp \N_\infty$, we have $s_M \simeq L$ for some real $L$. Suppose it is not convergent. Therefore, there must exists $N \in \hyp \N_\infty$ such that $s_N \not \simeq L$. However since $(s_n)$ is bounded, we have $s_N$ to be limited and hence infinitely close to a real number ie its shadow. But if $\sh(s_N) \not = L$, clearly $\sh(s_N)$ is a cluster point which contradicts the assumption that there was an unique cluster point. \\
    Consider the sequence $\la s \ra=(0,1,0,2,0,3,\cdots)$. Clearly this sequence has only one cluster point and is unbounded but it does not converges to any real number.
  \end{proof}
\newpage
  \begin{exercise}
    (Section 6.8, I)
    Prove, by nonstandard reasoning, that both the limit superior and limit inferior are cluster points of the sequence $s$.
  \end{exercise}
  \begin{proof}
    Let $L=\limsup s$ and $\ep \in \R^+$. Since $L$ is supremum of $C_s$, we know there exists $N \in \hyp \N_\infty$ such that $L-\ep <s_N \leq L$. Consider some arbitrary $n \in \N$. We have $(\exists m \in \hyp \N)(m>n \land L-\ep<s_m)$ to be true, and applying transfer on it, it follows $(\exists m \in \N)(m>n \land L-\ep<s_n)$ is true in $\R$. Since $\ep \in \R^+$ and $n \in \N$ was arbitrary, $(\forall \ep \in \R^+)(\forall n \in \N) (\exists m \in \N)(m>n \land L-\ep<s_m)$ is true in $\R$. Applying transfer we see $(\forall \ep \in \hyp \R^+) (\forall n \in \hyp \N)(\exists m \in \hyp \N)(m>n \land L-\ep<s_n)$ holds in $\hyp \R$. Picking $\ep$ as some infinitesimal and $n$ as unlimted, we follow there is some $m \in \hyp \N$ making $m>n$ and $L-e<s_m$ true, that is to say, there is unlimited $m$ such that $L<s_m+e$. Therefore, it follows $L \leq \sh(s_m)$ for an unlimited $m$. However, since $L$ is upper bound of $C_s$, we must have $L=\sh(s_m)$ making it a cluster point. We cab argue about $\lim \inf$ in a similar fashion.
  \end{proof}
  \begin{exercise}
    (Section 6.9, I)
    Formulate the definition of the limit inferior of an arbitrary real valued sequence.
  \end{exercise}
  \begin{answer} $ $ \\
    \textbf{Proposition} A real number $L$ is equal to $\liminf s$ if and only if \\
    (1) $L<s_n$ or $L \simeq s_n$ for all unlimited $n$; and \\
    (2) $s_n \simeq L$ fdor atleast one unlimited $n$. \\
  \begin{proof} 
    The condition (1) holds iff $L \leq \sh(s_n)$. Thus, it implies $L$ is lower bound of set of cluster points, $C_s$. The condition (2) asserts that $L$ is a cluster point i.e. $L \in C_s$. Together they imply that $L$ is infimum (since it's the smallest element) of the set. Hence, it is equal to $\liminf s$.  
  \end{proof}
  \end{answer}
  
  \begin{exercise}
    Formulate and prove theorems about the limit inferior of a bounded sequence that correspond to theorems 6.8.4 and 6.8.5.
  \end{exercise} 
  \begin{answer} $ $ \\
    \textbf{Proposition-1.} If $s$ is a bounded real valued sequence with limit inferior, $\liminf$, then for any positive real $\ep$: \\
    (1) Some standard tail of $s$ has all its terms bigger than $\liminf-\ep$ i.e. $\liminf-\ep<s_n$ for all but finitely many $n \in \N$. \\
    (2) $s_n<\liminf +\ep$ for infinitely many $n \in \N$.
    \begin{proof}
      Let $L=\liminf s$. Consider $\ep \in \R^+$. First we prove (1). For any unlimited $N$, we have $L<s_N$ or $L \simeq s_N$. Thus, $L-\ep<s_N$ holds for any unlimited $N$. Trasnferring $(\exists n \in \hyp \N)(\forall m \in \hyp \N)(m \geq n \land L-\ep<s_m)$, we have $(\exists n \in \N)(\forall m \in \N)(m>n \land L-\ep<s_m)$ to be true in $\R$ but that simply means $L-\ep<s_m$ is true for all but finitely many $n$. We now prove (2). Let $m \in \N$. We know that $L$ is infinitely close to $s_N$ for some unlimited $N$. Thus, $(\exists n \in \hyp \N)(n>m \land s_n<L+\ep)$ is true in $\hyp \R$. Transfering it, we have $(\exists n \in \N)(n>m \land s_n<L+\ep)$ to be true in $\R$. However since $m \in \N$ was arbitrary, it follows that for any $m \in \N$ there exists $n \in \N$ such that $n>m$ and $s_n<L+e$ satisfying the proposition.
    \end{proof}

    \textbf{Proposition-2.} For any bounded real valued sequence $s$, $$\liminf_{n \arr \infty} s_n=\lim_{n \arr \infty} I_n=\lim_{n \arr \infty} \left(\inf_{m \geq n} s_m \right)$$ where $I_n=\inf_{m \geq n} s_m$.
    \begin{proof}
      First we will prove that $\liminf$ is upper bound of $S$. Take an extended term $s_N$ infinitely close to the cluster point $\liminf$. For any $m \in \N$, we have $I_m \leq s_n$ for all limited $n \geq m$, and hence it holds for all hypernatural $n \geq m$ by transfer. In particular, $I_m \leq s_N$, as $\liminf \simeq s_N$, we have $I_m \leq \liminf$. \\
      Let $L=\lim S_n$. Then clearly $L$ is supremum of $S$ since $S$ is nondecreasing. However, as we showed $\liminf$ is an upper bound, we must have $L \leq \liminf$. We will now show that $L< \liminf$ is impossibility. Suppose $L< \liminf$ and consider some $\ep \in \R^+$. By proposition-1, we know that there exists a $n \in \N$ such that standard tail beyond $n$ is bounded below by $\liminf-\ep$.
    \end{proof}

  \end{answer}
  

\end{document}

